\section{The Dining Cryptographers Problem}

The idea of anonymous message broadcasting is explained most easily using an
example [a theoretical example??reword].  If three cryptographers employed by
the NSA were to go out for dinner at a restaurant, then when it came time to
pay for the meal they would face the decision of who was to pay.  The
cryptographers decide amongst themselves that either the NSA or one of them
will pay for the meal, but they only want to be able to tell whether one of the
cryptographers or their employer paid, not which of the cryptographers paid (if
any).  A simple protocol involving coin flipping can be used to solve this
problem, and is the basis for more complicated message broadcast protocols,
which will be described in a later section. \\

The protocol to determine who paid the meal anonymously works by having each
cryptographer secretly flip a coin with the cryptographer sitting to their right.  This
results in each cryptographer being able to see two coins, which will show
either the same side (e.g. heads and heads), or different sides (e.g. heads and
tails).  Each cryptographer then decides if they want to pay for the meal or
not.  If they aren't paying for the meal, they simply state whether the two
coins they can see are the same or different.  If they decide to pay for the
meal, then they say the opposite of what they actually see.  In the end, this
means that if an odd number of 'different's are announced, one of the
cryptographers paid, while if an even number of 'different's are mentioned, the
NSA paid for the meal. \\

\section{Expanding the Dining Cryptographers}

The solution to the Dining Cryptographers problem is not only elegant, but it
can be scaled to many more users, and converted into a more general form, where
each participant in the system is represented as a node in a graph, and the
shared coin flip becomes a key shared between two nodes in the graph i.e. an
edge.  To expand the idea further, it is generalized to form a system where a
number of participants can share messages with each other without the messages
creator being identifiable, resulting in a system for anonymous message
broadcast. \\

While the generalized system can allow edges between all participants, the
basic ring structure of the Dining Cryptographers Problem was maintained.
This was because the implementation of the anonymous message broadcast system
used a client server architecture, where the server is responsible for handing
keys out to all parties, then requesting and retrieving individual messages
from the clients, and the extra overhead required to allow clients to shares
keys with every single other client seemed more complex than necessary for an
initial implementation of a broadcast system. \\
